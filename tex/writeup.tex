\documentclass{article}
\usepackage[utf8]{inputenc}
\usepackage{amsmath}
\usepackage{amssymb}
\usepackage{graphicx}
\usepackage[margin = 1 in]{geometry}



\title{CS5220 Project 2: Shallow Water Equations}
\author{Nick Cebry, Jiahao Li, Max Ruth}
\date{October 5, 2020}

\begin{document}

\maketitle

\section{Introduction}
This could probably be just a paragraph summarizing the assignment

\section{The Algorithm}
The general process:
\begin{enumerate}
	\item Used MPI
	\item Transfer Boundary information to left and right neighbors, the top and bottom (transferring the corners to top and bottom)
	\item Transfer boundary information every $\tau$ steps
	\item Transfer time step information every $\tau$ steps as well (This is what is being used in the model anyway)
\end{enumerate}
The model
\begin{enumerate}
	\item Four different contributions to batches of time steps
	\item Give predictions for how strong scaling, weak scaling, and optimal time batching look like.
\end{enumerate}

\section{Scaling and Profiling Results}
Figure this out!



\section{Conclusion}
What we would add for next time:
\begin{enumerate}
	\item More careful cache performance
	\item Tuning number of ghost cells to block size
	\item Different sized domains/initial conditions
	\item Deal with the fact that not all processors are the same
	\item Think about the tradeoff between a conservative time step and communicating every step (how uneven are the time steps really?)
\end{enumerate}


\end{document}